\documentclass[letter, 11pt]{article}
%% ================================
%% Packages =======================
\usepackage[utf8]{inputenc}      %%
\usepackage[T1]{fontenc}         %%
\usepackage{lmodern}             %%
\usepackage[spanish]{babel}      %%
\decimalpoint                    %%
\usepackage{fullpage}            %%
\usepackage{fancyhdr}            %%
\usepackage{graphicx}            %%
\usepackage{amsmath}             %%
\usepackage{color}               %%
\usepackage{mdframed}            %%
\usepackage[colorlinks]{hyperref}%%
%% ================================
%% ================================

%% ================================
%% Page size/borders config =======
\setlength{\oddsidemargin}{0in}  %%
\setlength{\evensidemargin}{0in} %%
\setlength{\marginparwidth}{0in} %%
\setlength{\marginparsep}{0in}   %%
\setlength{\voffset}{-0.5in}     %%
\setlength{\hoffset}{0in}        %%
\setlength{\topmargin}{0in}      %%
\setlength{\headheight}{54pt}    %%
\setlength{\headsep}{1em}        %%
\setlength{\textheight}{8.5in}   %%
\setlength{\footskip}{0.5in}     %%
%% ================================
%% ================================

%% =============================================================
%% Headers setup, environments, colors, etc.
%%
%% Header ------------------------------------------------------
\fancypagestyle{firstpage}
{
  \fancyhf{}
  \lhead{\includegraphics[height=4.5em]{LogoDFI.jpg}}
  \rhead{FI3104-1 \semestre\\
         Métodos Numéricos para la Ciencia e Ingeniería\\
         Prof.: \profesor}
  \fancyfoot[C]{\thepage}
}

\pagestyle{plain}
\fancyhf{}
\fancyfoot[C]{\thepage}
%% -------------------------------------------------------------
%% Environments -------------------------------------------------
\newmdenv[
  linecolor=gray,
  fontcolor=gray,
  linewidth=0.2em,
  topline=false,
  bottomline=false,
  rightline=false,
  skipabove=\topsep
  skipbelow=\topsep,
]{ayuda}
%% -------------------------------------------------------------
%% Colors ------------------------------------------------------
\definecolor{gray}{rgb}{0.5, 0.5, 0.5}
%% -------------------------------------------------------------
%% Aliases ------------------------------------------------------
\newcommand{\scipy}{\texttt{scipy}}
%% -------------------------------------------------------------
%% =============================================================
%% =============================================================================
%% CONFIGURACION DEL DOCUMENTO =================================================
%% Llenar con la información pertinente al curso y la tarea
%%
\newcommand{\tareanro}{XX}
\newcommand{\fechaentrega}{XX/XX/XXXX XX:XX hrs}
\newcommand{\semestre}{201XB}
\newcommand{\profesor}{Valentino González}
%% =============================================================================
%% =============================================================================


\begin{document}
\thispagestyle{firstpage}

\begin{center}
  {\uppercase{\LARGE \bf Tarea \tareanro}}\\
  Fecha de entrega: \fechaentrega
\end{center}


%% =============================================================================
%% ENUNCIADO ===================================================================
\noindent{\large \bf Problema}

{\it Un problema común en estadística es el de calcular la función percentil,
que corresponde a la inversa de la función distribución acumulada.}

Supongamos que $x$ es una variable aleatoria, es decir, un número elegido al
azar, pero siguiendo una distribución de probabilidad dada. En este caso
asumiremos que la distribución de probabilidad de $x$ es gausiana con
parámetros $\mu=0$ y $\sigma=1$. Esto significa que, al tomar un número de la
distribución, la probabilidad de que ese número sea mayor que un determinado
valor $a$, se puede calcular como:

$$ p(x>a) = \int_{a}^{\infty} \frac{1}{\sqrt{2\pi}} \exp\left({\frac{-y^2}{2}}\right) dy $$

A menudo, la pregunta que necesitamos responder es: ¿Qué valor de $a$ debemos
escoger para asegurarnos de que una variable aleatoria sacada de dicha
distribución no sea {\it casi nunca} mayor que $a$? Donde {\it casi nunca} en
este caso significa, {\it menos del 5\% de las veces}.

Este problema se puede plantear como la siguiente ecuación para $a$:

\begin{equation}
0.05 = \int_{a}^{\infty} \frac{1}{\sqrt{2\pi}} \exp\left({\frac{-y^2}{2}}\right) dy
\end{equation}

Resuelva la ecuación (1) numéricamente e interprete su resultado.

\begin{ayuda}
  Para calcular la integral deberá hacer un cambio de variable, se recomienda:

  $$ u = 1/y$$
\end{ayuda}

%% FIN ENUNCIADO ===============================================================
%% =============================================================================

\vspace{1em}
\noindent{\bf Otras instrucciones importantes.}
\begin{itemize}

    \item Utilice \texttt{git} durante el desarrollo de la tarea para mantener
        un historial de los cambios realizados. La siguiente
        \href{https://education.github.com/git-cheat-sheet-education.pdf}{cheat
        sheet} le puede ser útil. Esto no será evaluado esta vez pero
        evaluaremos el use efectivo de \texttt{git} en el futuro, así que
        empiece a usarlo.

    \item La tarea se entrega como un \texttt{push} simple a su repositorio
        privado. El \texttt{push} debe incluir todos los códigos usados además
        de su informe.

    \item El informe debe ser entregado en formato \texttt{pdf}, este debe ser
      claro sin información ni de más ni de menos. Esto es importante, no
      escriba de más, esto no mejorara su nota sino que al contrario. 3 páginas
      es un largo razonable para la presente tarea.  Asegúrese de utilizar
      figuras efectivas y/o tablas para resumir sus resultados. Revise su
      ortografía.

    \item No olvide indicar su RUT en el informe.

\end{itemize}


\end{document}
